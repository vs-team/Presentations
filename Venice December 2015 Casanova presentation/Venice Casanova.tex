\documentclass{beamer}
\usetheme[hideothersubsections]{HRTheme}
\usepackage{beamerthemeHRTheme}
\usepackage[utf8]{inputenc}
\usepackage{graphicx}

\usepackage{multirow}
\usepackage{MnSymbol}

\title{Introducing Casanova 2, a pragmatic domain specific language for game development}

\author{\textbf{Mohamed Abbadi}}


\institute{\textbf{Ca'Foscari University}\\ 
Venice, Italy\\
\textbf{Tilburg University}\\
Tilburg, Netherlands}

\date{}

\begin{document}
\maketitle

\SlideSection{Introduction}
\SlideSubSection{Lecture topics}
\begin{slide}{
\item The relevance of games and beyond
\item A brief introduction to game architectures
\item Are we missing a language?
\item Casanova as a first step towards a solution
\item Current state of progress
}\end{slide}

\SlideSection{The relevance of games and beyond}
\SlideSubSection{Background}
\begin{slide}{
\item Games are a huge market (\textbf{b}illions of worth, countless hours spent)
\item Changing the entertainment landscape, overtaking music and movies
\item Large industry, with lots of technical innovation and inspiring work
}\end{slide}

\SlideSubSection{Beyond entertainment}
\begin{slide}{
\item The success of games is inspiring more and more people
\item What if we can make other stuff as engaging?
\item And so, \textit{serious games} are born
}\end{slide}

\SlideSubSection{Serious games}
\begin{slide}{
\item Using game mechanics to bring serious applications to life
\item Simulators, training applications, rehabilitation, school, and much more
\item Countless applications could benefit from games, accelerating progress and culture from many angles
}\end{slide}

\SlideSubSection{What is stopping us?}
\begin{slide}{
\item Great, so where is my ``learn math by blowing zombies up'' game?
\item Serious games makers do not have huge resources
\item Here comes the problem: \textbf{games are very hard and expensive to build}
\pause
\item \textbf{This heavily impacts the viability of games in such innovative applications}
}\end{slide}

\SlideSubSection{Our take}
\begin{slide}{
\item Our goal is to study and optimize the technical aspects of game development
\item We look for complexity in the process and try to automate it
\item We focus on the \textit{programming languages} side of the equation\footnote{There are many others!}
}\end{slide}

\SlideSection{A brief introduction to game architectures}
\SlideSubSection{The game loop - a source of complexity}
\begin{slide}{
\item Games need to run in real time
\item Frames must be sent to the screen at 30/60Hz
\item It's all a big loop!
}\end{slide}

\begin{frame}[fragile]{A very simplified game architecture}
\begin{lstlisting}
state = GameState()
while Running:
  state = update(state, dt)
  state.Draw()
\end{lstlisting}
\end{frame}

\SlideSubSection{The game loop - a source of complexity}
\begin{slide}{
\item \texttt{update} covers the dynamics of the game
\item The game is a \textit{hybrid system}
\item The continuous parts are simple: just integrate numerically (\texttt{P = P + V * dt})
\item The discrete parts are less simple: large concurrent state machines!
}\end{slide}

\begin{frame}[fragile]{Discrete architectures}
\begin{codewithblock}{\pause \item That's not how we think!}
switch action:
  case RUNNING:
    ...
  case FIGHTING:
    ...
  ...
\end{codewithblock}
\end{frame}

\SlideSection{Are we missing a language?}
\SlideSubSection{Idea}
\begin{slide}{
\item Most of what makes a game a game is the discrete part of the dynamics
\item Create interesting patterns that introduce \textit{controlled} variation
\item These dynamics define game logic, AI, and more
}\end{slide}

\begin{frame}[fragile]{Specification of discrete dynamics}
\begin{codewithblock}{\pause \item Now imagine the ``big switch'' to implement this}
repeat
  for c in checkpoints:
    velocity = normalize(c - position)
    wait (distance(position, c) < 1m)
    animation = look around
    wait 3s
\end{codewithblock}
\end{frame}

\SlideSection{Abstraction mismatch}
\begin{slide}{
\item The description of these dymamics is usually difficult to find back in code
\item In this translation we lose a lot of the original intention
\item It is harder to maintain, extend, and experiment with code
}\end{slide}

\SlideSection{Existing solutions}
\begin{slide}{
\item There are possible solutions (threads, strategy pattern, monadic coroutines, etc.)
\item They all suffer from performance drawbacks, and have lots of hidden complexities
}\end{slide}

\SlideSection{Casanova as a first step towards a solution}
\begin{slide}{
\item We will now present Casanova, a domain specific language (DSL) for game development
\item The language has time as a first class primitive
\item All computations are specified as transformations of the state
\item Continuous and discrete dynamics are both directly supported
}\end{slide}

\begin{frame}[fragile]{Continuous dynamics}
\begin{lstlisting}
entity Asteroid = {
  inherit UnityAsteroid
  Velocity : Vector3
  
  rule Position = Position + Velocity * dt

  Create() =
  {
    Velocity      = new Vector3(...)
    UnityAsteroid = UnityAsteroid.Instantiate(...)
  }
}
\end{lstlisting}
\end{frame}

\begin{frame}[fragile]{Discrete dynamics}
\begin{lstlisting}
worldEntity World = {
  inherit UnityBob

  rule Velocity, CurrentAnimation = 
    for c in Checkpoints do
      let dir0 = c - Position
      yield dir0, BobAnimation.Walk
      wait Vector3.Dot(dir0, c - Position) < 0.0f
      yield Vector3.zero, BobAnimation.Idle
      wait 1.0f

  Create() =
    {
      UnityBob = UnityBob.Find()
    }
}
\end{lstlisting}
\end{frame}

\SlideSubSection{Ok, surely it does not work?}
\begin{slide}{
\item We have been eating our own dog food quite extensively :)
\item We have built a lot of samples and smaller games with Casanova
\pause
\item The good: quite simply, \textbf{it works}
\item The bad: has a bit of a learning curve
}\end{slide}

\SlideSubSection{Demo}
\begin{pictureSlide}
{someMoreGames.png}{8}
\end{pictureSlide}

\begin{picturetextslide}{someGames.png}{8}{
\item \url{www.youtube.com/watch?v=_dE3EQ_aPbE&sns=em}}
\end{picturetextslide}


%DESIGN and IMPLEMENTATION
\SlideSection{Casanova design}
\SlideSubSection{Syntax}
\begin{frame}[fragile]{Syntax}
\tiny	
\begin{lstlisting}[showstringspaces=false]
<Program> ::=
  <moduleStatement> {<openStatement>}
  <worldDecl> {<entityDecl>}

<moduleStatement> ::= module id
<openStatemnt>    ::= open id
<worldDecl>    ::= world id ["("<formals>")"] =
                   <worldOrEntityDecl>
<entityDecl>   ::= entity id ["("<formals>")"] =
                   <worldOrEntityDecl>
<worldOrEntityDecl> ::= "{" <entityBlock> "}"
<entityBlock>  ::= {<fieldDecl>} {<ruleDecl>}
<create>
<create> ::= Create "(" {<formals>} ") = <expr>
<formals>   ::= id [":" <type>] {"," <formals>}
<fieldDecl> ::= id [":" <type>]
<ruleDecl>  ::= rule id {"," id} "=" <expr>
<type>      ::= int |boolean  |float |Vector2
                |Vector3 |string |char
                |list "<" <type> ">" |<generic>
                |<type> "[" "]" |id
<generic>     ::= "'" id
<expr> ::= ...(* typical expressions : let, if,
                 for , while , new, etc. *)
           | wait (<arithExpr> | <boolExpr>)
           | yield | <arithExpr> | <boolExpr>
           | <literal> | <queryExpr> | <seq>
<seq>        ::= <expr> <expr>
<arithExpr>  ::= ...//arithmetic expressions
<boolExpr>   ::= ...//boolean expressions
<literal>    ::= ...//strings , numbers
<queryExpr>  ::= ...//query expressions
\end{lstlisting}
\end{frame}

\SlideSubSection{Semantics - re-write based}
\begin{frame}[fragile]{Semantics - re-write based}
\tiny	
\begin{lstlisting}[showstringspaces=false]
E = { Field$_1$ = f$_1$; $\dots$; Field$_n$ = f$_n$;
Rule$_1$ = r$_1$; $\dots$; Rule$_m$ = r$_m$ }

tick(e:E, dt) =
  { Field$_1$=tick(f$_1^m$, dt); $\dots$; Field$_n$=tick(f$_n^m$, dt);
    Rule$_1$=r$_1'$; $\dots$; Rule$_m$=r$_m'$ }
  where
    f$_1^m$, $\dots$, f$_n^m$, r$_m'$ = step(f$_1^{m-1}$, $\dots$, f$_n^{m-1}$, r$_m$)
    .
    .
    f$_1^1$, $\dots$, f$_n^1$, r$_1'$ = step(f$_1$, $\dots$, f$_n$, r$_1$)


step(f$_1$, $\dots$, f$_n$, {let x = y in r$'$}) =
step(f$_1$, $\dots$, f$_n$, r$'$[x:=y])

step(f$_1$, $\dots$, f$_n$, {if x then r$'$ else r$''$; r$'''$})
  when (x = true) = step(f$_1$, $\dots$, f$_n$, {r$'$; r$'''$})

step(f$_1$, $\dots$, f$_n$, {if x then r$'$ else r$''$; r$'''$})
  when (x = false) = step(f$_1$, $\dots$, f$_n$, {r$''$; r$'''$})
.
.
step(f$_1$, $\dots$, f$_n$, {for x in [] do r$'$; r$''$})
  step(f$_1$, $\dots$, f$_n$, r$''$)
\end{lstlisting}
\end{frame}

\SlideSection{Implementation}
\SlideSubSection{Compiler}
\begin{slide}{
\item Casanova has its own compiler 
\item The first version used the open F\# compiler as back-end
\begin{itemize}
\item Pros: Fast to set up
\item Cons: lack of control over code transformations
\end{itemize}
\item The current version is \textit{layer based}. We wrote our own back-end, which now produces C\#
\begin{itemize}
	\item Pros: complete control, C\# can be run on several OS and tools
	\item Cons: complex to build, scalability issues
\end{itemize}
}\end{slide}


\begin{slide}{
\item Control over layers transformation and domain syntax and semantics allows code optimization
\item Generated code uses low level constructs to achieve high performance
\item[] 
\hspace{-1cm}
\begin{minipage}[t]{.4\textwidth}
\begin{table}[!t]
\tiny
\caption{Cnv performance}
\begin{tabular}{|c||c|}
	\hline
	Language & Time per frame\\
	\hline
	Casanova & 0.07ms\\
	\hline
	C\# & 0.12ms\\
	\hline
	JavaScript & 24.07ms \\
	\hline
	Lua & 20.90ms\\
	\hline
	Python & 20.15ms \\
	\hline
\end{tabular}
\end{table}
\end{minipage}% 
\begin{minipage}[t]{.6\textwidth}

\begin{table}[!ht]
\tiny
\caption{Cnv vs Cnv opt}
\begin{tabular}{ @{}|c|c|c|c|@{} }
\hline
Platform & Language & Optimized & Performance\\ \hline
\multirow{3}{*}{Monogame}
& Casanova & No & 0.0159 ms\\
& Casanova & Yes & 0.0098 ms\\
& C\# &   - & 0.0147 ms\\ \hline
\multirow{3}{*}{Unity3D}
& Casanova & No & 0.0257 ms\\
& Casanova & Yes & 0.0085 ms\\
& C\# &   - & 0.1642 ms\\ \hline
\hline
\end{tabular}
\end{table}
\end{minipage}
}\end{slide}

\SlideSubSection{What is Casanova?}
\begin{slide}{
\item Language features
\begin{itemize}
	\item It is a domain language
	\item Hybrid (OO, Functional, Declarative)
	\item Statically compiled
	\item Performance oriented
\end{itemize}
\item Compatibility (so far)
\begin{itemize}
\item Unity (web, mobile, desktop)
\item Lego Mindstorm V3
\item Monogame
\item ...
\end{itemize}
}\end{slide}


\SlideSection{Usability}
\SlideSubSection{Ease of use}
%USABILITY
\begin{slide}{
\item An important part of my research
\item Early results suggested us to compile Cnv into C\# and to use Unity as rendering tool
\item Cnv has been used for a year intensively by a group of 6 internship students 
\item 3 already done workshops on Casanova
\item[]
\begin{itemize}
\item \includegraphics[width=1cm]{game_on.png},  latest workshop run at UoA 
\end{itemize}
\item Usability goals
\begin{itemize}
\item Perfect the quality of our tools
\item Possibly adapt/introduce new features
\end{itemize}
}\end{slide}


%WRITTEN CONTRIBUTION
\SlideSection{Written conributions}
\SlideSubSection{Research progress}
\begin{pictureSlideWithCaption}
{research_progress.png}{8}{Research progress}
\end{pictureSlideWithCaption}
%publications
% already published
% in progress
% book chapter

%USABILITY
\begin{slide}{
\item Papers
\begin{itemize}
\item Title: Resource Entity Action: A Generalized Design Pattern for RTS games (Springer)
\item Title: Casanova. A simple, high-performance language for game development (Springer)
\item Title: High performance encapsulation in Casanova 2 (IEEE)
\item Title: Making RTS games in Casanova (not submitted yet)
\end{itemize}
\item Book chapter
\begin{itemize}
	\item Title: Serious Games (Springer)
\end{itemize}
\item \textbf{LOTS} of Samples, workshop materials, and a collection of on-line tutorials
}\end{slide}

\SlideSection{Conclusion}
\SlideSubSection{Conclusion}
\begin{slide}{
\item Game dynamics are hard to express
\item By defining programming languages that support the building blocks of these dynamics, we can achieve higher productivity
\item Games have a huge potential, and with Casanova we hope to unlock it
\item[]
\item \url{https://github.com/vs-team/casanova-mk2/wiki}
}\end{slide}

\begin{thankyou}
\end{thankyou}

\end{document}

\begin{slide}{
\item ...
}\end{slide}

\begin{frame}[fragile]
\begin{lstlisting}
...
\end{lstlisting}
\end{frame}

\begin{frame}[fragile]
\begin{codewithblock}{\item $x^y + \sum_i i^2$}
if x > y then 
  print x 
else 
  print y
\end{codewithblock}
\end{frame}
