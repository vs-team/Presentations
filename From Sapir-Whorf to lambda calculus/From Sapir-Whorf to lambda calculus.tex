\documentclass{beamer}
\usetheme[hideothersubsections]{HRTheme}
\usepackage{beamerthemeHRTheme}
\usepackage[utf8]{inputenc}
\usepackage{graphicx}
\usepackage[space]{grffile}

\title{Sapir-Whorf and programming languages}

\author{Dr. G. Maggiore}

\institute{Hogeschool Rotterdam \\ 
Rotterdam, Netherlands}

\date{}

\begin{document}
\maketitle

\SlideSection{Introduction}
\SlideSubSection{Lecture topics}
\begin{slide}{
\item Linguistic relativity in natural languages
\item From natural to artificial (programming) languages
\item Syntax, semantics, and translatability (Turing-equivalence)
\item Linguistic relativity in artificial languages
\item (Designing artificial languages)
}\end{slide}

\SlideSubSection{Problem discussion}
\begin{slide}{
\item Reasoning about languages, expressive power, and potential for translation
\item Using artificial, mathematical languages to quantify these concepts
}\end{slide}

\SlideSubSection{Linguistic relativity in natural languages (\textit{Sapir–Whorf hypothesis})}
\begin{slide}{
\item Structure of a language affects world view and/or cognitive processes
\item Strong version \textbf{language determines thought}
\item Weak version \textbf{language influences thought}
}\end{slide}

\SlideSubSection{Relevance for philosophy, psychology, and linguistics}
\begin{slide}{
\item Are human psychological faculties innate, or are they a result of learning?
\item Universalism: same biological construct makes culture irrelevant
\item Constructivism: human faculties and concepts come from socially constructed and learned categories
}\end{slide}

\begin{slide}{
\item Relation between language and thought
\item Universalism: thought as a form of innate internal speech
\item Constructivism: thought is learned while acquiring language
}\end{slide}

\begin{slide}{
\item Ultimately, it boils down to a deep question
\item Is there an objectively known/knowable reality, shared by all humans?
\item If so, is this reality objectively depicted with language constructs, and thus communicable?
\item If not, how do we bridge the gap between languages?
}\end{slide}

\SlideSubSection{Examples}
\begin{slide}{
\item In English, WATER is described with many possible words
\begin{itemize}
\item water as a LIQUID
\item water in the form of a large expanse (LAKE)
\item water as running in a large body or in a small body (RIVER and BROOK)
\item water in the form of RAIN, DEW, WAVE, and FOAM
\item ... 
\end{itemize}
}\end{slide}

\begin{slide}{
\item In Eskimo, SNOW is described with many possible words
\begin{itemize}
\item \textit{aput}, expressing SNOW ON THE GROUND
\item \textit{qana}, FALLING SNOW
\item \textit{piqsirpoq}, DRIFTING SNOW
\item \textit{qimuqsuq}, A SNOWDRIFT
\item ... 
\end{itemize}
}\end{slide}

\begin{slide}{
\item Guugu Yimithirr (Australian aboriginal language) only uses absolute directions when describing spatial relations
\item A person is \textit{north of the house}, not \textit{in front of the house}
\item Guugu Yimithirr speakers are better at navigating open terrain
\item English speakers are better at positioning objects relative to the speaker 
\begin{itemize}
\item For example consider describing a round table with forks to the right of the plate and knives to the left in Guugu Yimithirr
\end{itemize}
}\end{slide}

\SlideSubSection{Central question}
\begin{slide}{
\item \textbf{Does the choice of language make us more or less effective at experiencing, navigating, and communicating the world?}
}\end{slide}

\SlideSection{From natural to artificial (programming) languages}
\SlideSubSection{Introduction}
\begin{slide}{
\item Languages are not limited to natural ones
\item There are many artificial languages of daily use
\item Think of the one closest to you... \pause \textbf{in your pocket: your mobile phone GUI!}
}\end{slide}

\SlideSubSection{The computer-whisperer}
\begin{slide}{
\item Communication with machines happens through a variety of (programming) languages
\item Some very limited languages are simple visual tools and GUI's
\item Some very complex programming languages (PL's) have full blown syntaxes and semantics
}\end{slide}

\begin{slide}{
\item Modern computers seem to be capable of amazing approximations of complex human thoughts
\item Issue is, computers are \textbf{very fast at being very stupid}
\item PL's are complex, articulated, and hard to use
}\end{slide}

\begin{slide}{
\item Note the plural: language\textbf{S}
\item Why would there be multiple languages to talk to computers, which are the same? \pause Because of linguistic relativity!
\item It is empirically evident that 
\begin{itemize}
\item Different PL's have...
\item ...different expressive power...
\item ...in some specific domain(s)
\end{itemize}
}\end{slide}

\begin{slide}{
\item Different PL's have different expressive power
\item Assembly language
\begin{itemize}
\item Expresses any possible program
\item Is very fast
\item Is very verbose
\item Accepts many nonsensical programs
\item Works in terms of machine capabilities
\end{itemize}
\item Haskell
\begin{itemize}
\item Expresses only non-strict programs
\item Is very slow
\item Is very compact
\item Accepts very few nonsensical programs
\item Works in terms of categorical constructions
\end{itemize}
}\end{slide}

\SlideSection{Syntax, semantics, and Turing-completeness}
\SlideSubSection{Introduction}
\begin{slide}{
\item Let us now take a (shallow) dive in the core of PL's
\item We begin with the lambda calculus ($\lambda$-calculus)
}\end{slide}

\SlideSubSection{Syntax of the $\lambda$-calculus}
\begin{textslide}{
Valid programs are made up of
\begin{description}
\item[variables] $x$
\item[functions] $\lambda x.t$
\item[applications] $t\ s$
\end{description}
}\end{textslide}

\SlideSubSection{$\beta$ semantics of the $\lambda$-calculus}
\begin{slide}{
\item Semantics of the $\lambda$-calculus are based on rewriting
\item There is nothing but the language, transforming into itself
\item Complex concepts are thus \textit{unfolded} from abstracted descriptions
\item \textbf{Language is thought and thought is language}
}\end{slide}

\begin{slide}{
\item $(\lambda x.M) N \rightarrow_\beta M[x \mapsto N] $
\item \textbf{Meaning as transformation}
}\end{slide}

\SlideSubSection{Representation of Church/Peano numerals}
\begin{textslide}{
\begin{description}
\item[\texttt{0}] $\lambda s.\lambda z.z$
\item[\texttt{1}] $\lambda s.\lambda z.s\ z$
\item[\texttt{2}] $\lambda s.\lambda z.s(s\ z)$
\item[\texttt{N}] $\lambda s.\lambda z.s^N\ z$
\end{description}
}\end{textslide}

\begin{textslide}{
\begin{description}
\item[\texttt{PLUS}] $\lambda m.\lambda n.\lambda s.\lambda z.m\ s\ (n\ s\ z)$
\item[\texttt{MULT}] $\lambda m.\lambda n.m\ (PLUS\ n)\ 0$
\end{description}
}\end{textslide}

\SlideSubSection{Other languages}
\begin{slide}{
\item The $\lambda$-calculus is not the only possible core language
\item Combinatory logic is also another possibility\footnote{Just look at the shape of the languages, complete understanding is not needed!}
}\end{slide}

\SlideSubSection{Syntax and semantics of combinatory logic}
\begin{textslide}{
Valid programs are made up of
\begin{description}
\item[I] $Ix \rightarrow x$
\item[K] $Kxy \rightarrow x$
\item[S] $Sxyz \rightarrow xz(yz)$
\item[B] $Bxyz \rightarrow x(yz)$
\item[C] $Cxyz \rightarrow xzy$
\end{description}
}\end{textslide}

\SlideSubSection{Combinatory logic vs $\lambda$-calculus}
\begin{slide}{
\item Clearly two very different things
\item Can one do things the other cannot?
\pause
\item \textbf{No!} They are absolutely equivalent
\item We can translate back and forth without losing expressive power
}\end{slide}

\begin{pictureSlide}{translationLambdaCL.PNG}{10}
\end{pictureSlide}

\SlideSection{Linguistic relativity in artificial languages}
\SlideSubSection{A partial conclusion}
\begin{slide}{
\item Within the logical domain, there is no difference in expressive power
\item When a language becomes powerful enough, then any concept can be expressed in it
}\end{slide}

\begin{slide}{
\item A PL is \textbf{powerful enough} according to the Church-Turing hypothesis
\item Translate back and forth into the $\lambda$-calculus
\item No language was ever found\footnote{Yet.} that can express programs that the $\lambda$-calculus cannot
}\end{slide}

\SlideSubSection{Translatability is not equivalence?}
\begin{slide}{
\item Some programs in one language are very short in the other (reading overhead)
\item Some proofs/tests of correctness in one language are very compact in the other (verification overhead)
\item Some programs in one language are very fast in the other (runtime overhead)
\item ...
}\end{slide}

\begin{slide}{
\item Performance in writing, executing, and reasoning about programs make a lot of difference
\item A lot of progress\footnote{From self-driving cars to intelligent robots} hinges on our ability to write better software
\item Bugs do sometimes cost millions\footnote{Mariner I}, or even lives\footnote{Toyota accelerator}
}\end{slide}

\begin{slide}{
\item As we put machine intelligence in charge of cranes, cars, money, and private information, the centrality of programming and PL's becomes even more prominent
\item For this reason new PL's are designed almost constantly
\item Some of these are domain specific, others attempt to be a \textit{Jack of all trades}
}\end{slide}

\begin{slide}{
\item The Sapir-Whorf hypothesis remains, even in the PL's world
\item The choice of PL's changes the ability of a programmer to build things with it
\item We look for the abstract patterns of the previous generation of languages and turn them into syntactic and semantic constructions
}\end{slide}

\SlideSubSection{Candidates for language design elements}
\begin{slide}{
\item Time and concurrency
\item Image processing/machine learning
\item Totality of functions (termination)
\item ...
}\end{slide}

\SlideSection{Conclusion}
\SlideSubSection{Looking back}
\begin{slide}{
\item Available language constructs influence thoughts
\item PL's target machines instead of other humans, but have the same issues
\item Modern research in PL's is looking for \textit{Holy-Grail languages}
\item With the ideal PL, it might be possible to unlock new frontiers of computation and bring on new information revolutions
}\end{slide}

\begin{thankyou}
\end{thankyou}

\end{document}

\begin{slide}{
\item ...
}\end{slide}

\begin{frame}[fragile]
\begin{lstlisting}
...
\end{lstlisting}
\end{frame}
